\section{Hardware} 
\subsection{Framework}

We are going to use the standard robot as the development platform (hereinafter referred to as “Infantry”) which developed by DJI TECHNOLOGY CO.

Benefits of using Infantry includes high stability, low development costs.
Cons like low personality, bad mechanical design that would probably cuts off the LIDAR scanning range also exist.

The hardware parts mainly includes mechanical part and device part.

\subsection{Mechanical part}

The mechanical part mainly consist of two modules; chassis module and gimbal module.

\subsubsection{Chassis module}

By using Mecanum wheels, Infantry are enable active omnidirectional movement.

Four 3508 Brushless DC gear motor with governors provide power to push the robot and payload.

We will be using RoboMaster Development Board Type A(STM32F427IIH6) as main control unit.

\subsubsection{Gimbal module}

Two dimensions gimbal drive by GM6020 DC brushless motor.

loading system drive by M2006 P36 Brushless DC Gear Motor.

Firing system drive by DJI Snail 2305 Racing motor.

We will be using RoboMaster Development Board Type A(STM32F427IIH6) as main control unit.

\subsection{Device part}
\subsubsection{Slave computer}

Two RM development board type A will be using as slave computer on chassis module and gimbal module separately.

It features various ports for further development such as UART port and CAN port that could allow it communicate with master computer, PWM port that could drive servo and brushless motors, and it also supports UWB localization and SDK development.

Multiple protection mechanisms including a built-in ESD on PWM interfaces prevent reverse connection, over-voltage and over-current.

\subsubsection{Host computer}
So far we have two choices for master computer, we could either simply buying the Nvidia TX2, it is expensive but will save a lots of time or just DIY an IPC, which will be obviously cheaper than the previous choice but time-consuming.

For the first choice, Jetson TX2 is a tiny little board built around Nvidia Pascal-family GPU with 256 CUDA cores which means faster speed on matrix multiplication.
The CPU complex consists of a Quad core A57 ARM processor connected to a dual core Denver processor, way slower than the I7, but it is ok to use.

For the DIY IPC choice, we would using the chassis with dimension of 18.5X4.5X19.75CM(W*H*D) to meet the Infantry mechanical space where chassis are going to be locate.

After all, we are going to remain both choice of the master computer, simply because a higher CPU performance is needed for ROS, so that it would be better choice if we developing on IPC platform and then transfer it to the TX2 after the developed.

\subsubsection{Camera}

Based on our previous experiences, it is highly import to have a camera that can reduce motion blur.
What’s more, the parameter adjustment is also one of the necessary functions.

And the convenience.
As mentioned above we finally chosen a camera that supports USB3.0, parameters adjustment and high frame rate for motion blur reduction.
The resolution of the camera is not that satisfactory, with only 640x480 resolution rate, but the competition does not requires the resolution rate that much.
It is quite enough to get image features such as amour light bar on this resolution rate. In short, the camera we have chosen is quite enough for the competition.

\subsubsection{LIDAR}

LIDAR is an extremely important device for localization and navigation.

Since the LIDAR works like shine a small light at a surface and measure the time it takes to return to its source, Therefore the adaptive scanning frequency and range sample frequency would be the high priorities for choosing a LIDAR.

The LIDAR we are going to use is the G4 LIDAR which developed by YDLIDAR CO, with range sample frequency about 9000hz, scanning range around 16m and maximum 12Hz adaptive scanning frequency.

The reason why we are choosing this LIDAR is because it has the highest adaptive scanning frequency and range sample frequency among same type of product, and the price is also quite reasonable. 

\subsubsection{Ultrasonic sensor}

Ultrasonic sensor will be using as a auxiliary positioning device of compensation and calibration for LiDAR positioning during the robot reloading process.

HY-SRF05 ultrasonic sensor can work from 2cm to 3m with 15 degrees measuring angle and ranging accuracy is reaching no more than 3mm.
Since it is a relatively mature product on the market therefore we chose it as auxiliary positioning device.

\pagebreak